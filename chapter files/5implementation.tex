Now that the light potentials have been measured, we elaborate in this chapter how it was implemented in an real-world experiment on ultra-cold atoms, initially using Rb atoms.
While our group has a lot of experience on Rb magneto-optical traps, because of the use of a glass cell and microscope objectives, we built a novel apparatus almost from scratch, apart from the laser system which we used from the previous setup.
The construction of the vacuum and atom source part can be found in \cref{sec:VacuumAtom}.
The laser system is explained in \cref{sec:LaserSystem}.
Images of the MOT are displayed in \cref{sec:MOTresult}.
We finish up with how overlap this MOT with our tweezer array (\cref{sec:Tweezers}) and how to image the tweezers (\cref{sec:TweezerImaging}).

\section{Vacuum and Atom Source}\label{sec:VacuumAtom}

The vacuum and atom source of the setup was designed with compactness in mind. 
The core of the experiment is the glass cell (30 mm outer and 4.0 mm wall thickness, optically contacted)\footnote{The glass cell was provided to us by our collaborators from the University of Amsterdam.}.

The microscope objectives are place in the vertical direction.
The coils producing the magnetic field gradient should also be placed near the cell. 
To still have room for all 6 MOT beams, we use a design similar to the \textit{Endres} group.
The idea is that two MOT beams and their retro-reflection pass the cell under a 60 degree angle, leaving room for the microscope objectives. 
The last MOT beam is sent through the hole of the anti-Helmholtz magnetic field coils.
To allow for placement of optics in the vertical direction, a vertical was used. 
In \cref{fig:VacuumSetup} the vertical breadboard is shown along with the vertical MOT beams and the magnetic field coils. 
Also the bottom objective is shown. 
The center of the glass cell is positioned 110 mm away from the vertical breadboard and 300 mm from the optical table. 
The housing for the magnetic field coils was designed by Deon Janse van Renseburg and Rick van Herk and is shown in \cref{fig:Coils} in \cref{subsec:Overlap}.

A real life picture of the glass cell is shown in \cref{fig:Chamber}.
It is pumped down to the ultra-high vacuum regime, which is need to minimize collisions with the background gas and improve the lifetime of the atoms in the tweezers.
To pump the cell down, the glass cell is connected to a vacuum chamber (\cref{fig:VacuumSetup}, left).
To allow for more room around the glass cell, in between the vacuum vessel and the glass cell an extension tube is added.
We attach a turbomolecular pump to the chamber using a valve (not shown in the figure) and pump down to $\sim 10^{8}$ mbar.
This pressure was measured using a pressure Gauge (\cref{fig:Chamber}).
To reach a better vacuum, the system was baked for a duration of two weeks at 130${}^{\circ}$C by Rik van Herk \cite{Herk2022}.

As atom source, a Rb dispensers\footnote{SAES Getters Alkali Metal Dispensers.} is used.
Only one dispenser is connected at a time, but we installed a triplet for redundancy.
The amount of Rb released can be controlled by the current running through the dispensers.
We typically run the dispensers at $\sim$ 5A.
The dispensers are mounted to the vacuum chamber using a CF40 adapter made by Eddy Rietman.
We first ran the dispensers for 30 minutes while leaving the turbo pump on to get rid of the oxidation layer on the dispenser.
This lead to a brief but significant increase in pressure, while we say no atomic fluorescence from Rb in the chamber.  
Next, we activate the non-evaporative getter\footnote{NEXTorr Z 100 NEG - ion combination pump.} by heating it to 500${}^{\circ}$C while pumping away its out gassing using the turbo pump. 
Finally, after closing the turbo valve and turning on the ion pump we reached a pressure of $\sim 2\times 10^{-10}$ mbar, as measured using the ion pump (the pressure gauge is not usable in this pressure regime).
When turning on the dispensers, the pressure briefly spikes to the $10^{-9}$ region but quickly returns back to the $10^{-10}$ regime.
One note: if this Rb machine is to be used for a longer time in the future, it would probably be wise to add a valve between the dispensers and the vacuum vessel, which allows for replacement of the dispensers without venting the entire chamber. 


\begin{figure}
	\begin{subfigure}{.5\linewidth}
		\flushleft
		\includegraphics[height=6.5cm]{figures/Vacuum.pdf}
		\caption{}
		\label{fig:VacuumSetup}
	\end{subfigure}
	\hfill
	\begin{subfigure}{.49\linewidth}
		\flushright
		\includegraphics[height=6.5cm]{figures/Chamber.jpg}
		\caption{}
		\label{fig:Chamber}
	\end{subfigure}
	\caption{\textsf{\textbf{a)}} CAD drawing of the vacuum vessel connected to the glass cell on the right. 
	Around the glass cell, 4 out of 6 MOT beams are shown along with the magnetic field coils and a single microscope objective on the bottom. 
	CAD drawing by Eddy Rietman.
    \textsf{\textbf{b)}} Picture of the vacuum components: top: ion/getter pump. Right: glass cell. Also connected to the chamber rom left to right: valve for turbo pump, triplet of Rb dispensers and pressure gauge.}
\end{figure}

\section{Laser System}\label{sec:LaserSystem}

For the laser system we recycled a significant part of the previous setup \cite{Reijnders2010}.
The atomic species we selected is Rb-85 because it is the most abundant isotope. 
a Rb-85 MOT requires two lasers: a trapping/cooling laser as well as a repump laser to recycle back atoms that end up in the wrong ground state (\cref{sec:PracticeRb}).
We used the same laser source for both of them. 

\subsection{Cooling/Trapping}

For the laser cooling and trapping beams we used the Toptica DLX110.
This laser should be able to produce $\sim 1$ W of power, but in current condition we typically output $\sim 500$ mW, which is already plenty of power for what we need. 
The repump light is produced by applying sidebands of fixed frequency spacing on the main trapping laser, more on this later.

\begin{figure}[t]
    \centering
    \includegraphics[width=\linewidth]{figures/RbLaserSetup.pdf}
    \caption{Laser system for the trapping and repump laser system for \textsuperscript{85}Rb.
    This system was built by \cite{Reijnders2010}.
    A section of laser light is split off to the modulation transfer spectroscopy (bottom left), after first double passing the lock \ac{AOM} (right). 
    Top: set of AOMs for the MOT and probe light.
    }
    \label{fig:RbLaserSetup}
\end{figure}

The laser is shown in \cref{fig:RbLaserSetup}.
After passing an optical isolator, a fraction of laser power is split off to the modulation transfer spectroscopy section after being offset-locked using the lock \ac{AOM} (right) set to $2\pi \times  84.5$ MHz\footnote{From now on, all frequencies are assumed to be angular and we omit the $2\pi$ term.}
Modulation transfer spectroscopy is described elsewhere \cite{McCarron2008,Reijnders2010}, but very briefly: the working principle is that by scanning the probe pulse using the \ac{EOM}, the absorption peak of $D_2$ is found using the Rb cell, which the laser is locked to.  
Contrary, the MOT AOM (top) is set to 80 MHz, which is $2 \times (80 - 84.5) = -9$ MHz from the resonance frequency, which is roughly equivalent to a detuning of $\delta = -1.5 \gamma$.
We typically use about $80$ mW of power for the MOT beams as measured after the \ac{AOM} in \cref{fig:RbLaserSetup}.
We have another AOM, which we use to make probe light, the light to induce fluorescence of the atoms when they are trapped in the optical tweezers.
The probe light is better described in \cref{sec:TweezerImaging}.

\begin{figure}[t]
    \centering
    \includegraphics[width=\textwidth]{figures/MOTupview.pdf}
    \caption{The laser for the MOT beams and probe beam are split to a vertical beam section (see \cref{fig:GlassCellSide}) as well as one horizontal beam, which travels through the anti-Helmholtz coil as shown here. 
    Both beams are expanded using a beam expander.
    This horizontal beam also contains the repump light using the \ac{EOM}.
    }
    \label{fig:GlassCellTop}
\end{figure}

\subsection{Repump}

Contrary to the previous setup, we do not use a separate repump laser. 
Instead, we used an \ac{EOM}.
The principle at play here is the electro-optic effect that can be used to modulate a beam of light. 
We drive the EOM with a modulation amplitude $\epsilon$ and frequency $\Omega$.
If the laser beam is described by amplitude $A$ and frequency 'carrier' frequency $\omega = \omega$ as $A = A_0 e^{i \omega_c t}$, after passing through an EOM the amplitude can be described as

\begin{equation}
    A(t) = A_0 e^{i \omega t}(1+i \epsilon \sin(\Omega t)
\end{equation}
where we assumed $\epsilon < 0$. In this case ($\epsilon <0$), most of the power is still contained in the carrier frequency component $\omega_c$, but a small fraction of power is contained in two first-order sidebands at $\omega_c \pm \Omega$.
We can use one of the sidebands as our repump laser. 
The other side-band is unused. 

As for the experimental details.
The EOM\footnote{7Qubig EO-Rb85-3K} was fed a $\Omega = 2915$ MHz signal provided by a harmonic synthesizer RF\footnote{DS instruments SG6000PRO} providing a signal of $-6.9$dbm.
This signal was amplified by a 45 dB amplifier\footnote{Minicircuits ZHL-16W-43+} which should provide some $\sim 10$\% of power in the first sidebands \cite{Rens2014}, which should be a sufficient amount of power for a repump laser. 
We did not measure the exact fraction of power in the sidebands because we do not have a sufficiently high-finesse wavemeter available, but the current parameters seemed to work satisfactory. 

\subsection{Distribution the Beams}

After passing through the AOMs, the MOT and probe light beams are directed further to the glass cell. 
This is shown in \cref{fig:GlassCellTop}.
Both beams are combined in a polarizing beam splitter cube, which also serves to split the MOT beams: one branch going to the horizontal section as shown in \cref{fig:GlassCellTop} as well as a beam that serves as the angled vertical MOT beams, as shown in \cref{fig:GlassCellSide}. 
Both beam paths are expanded to increase the overlap volume of the resulting MOT beams.
This was done using Galilean beam expanders. 
The lenses used for the horizontal beam are $f=-75$ mm and $f=400$ mm, which for a $\sim 0.9$ mm initial beam comes down to $\sim 0.9 \times 400/75 \sim 4.8$ mm.
The beam expander for the horizontal beam expands the $\sim0.9$ mm beam waist a factor of $400/75$ to $\sim5.4$ mm.
The angles beams are slightly bigger: because they are angled they need to be to still have a sufficiently big overlap volume. 

The half wave plates in front of the cube are set to only have probe light in the horizontal beam, while about 2/3 of the power for the MOT beams will travel to the vertical section. 
The repump light is only present in the horizontal beam: as a result of hitting the glass cell perpendicularly this beam has only a few per cent reflection losses, contrary to the vertical MOT beams. 
Because we only need repump light in one of the beams, it thus makes sense to do in this horizontal beam. 
The horizontal beam travels through the anti-Helmholtz coils as shown in \cref{fig:GlassCellTop}.
The mirror and $\lambdaup/4$ plate for retro-reflection are mounted directly on the vertical breadboard. 

Furthermore, the probe light is only directed to the horizontal beam path as well. 
This because the probe is on during the imaging sequences and the horizontal beam has considerably less reflections, meaning less background. 
Typically we run about an intensity of $I \sim I_{sat} =1.6$ mW/cm${}^2$ in the probe beam.
The detuning is a couple linewidths from the Stark shifted tweezers. 

\begin{figure}
    \centering
    \includegraphics[width=0.85\textwidth]{figures/MOTsideview.pdf}
    \caption{Side view on the glass cell, showing vertical MOT beams under a 60 degree angle, as well as the bottom microscope objective. 
    The top and side view MOT cameras are shown as well.}
    \label{fig:GlassCellSide}
\end{figure}


\section{Characterizing the MOT}\label{sec:MOTresult}

To image the MOT we position a CCD cameras\footnote{UEye UI-2230SE} looking at the MOT from the side, using the 5th and last available glass cell window. 
The camera is positioned in an $X-Y-Z$ translation stage and features a 780 nm band-pass filter. 
This camera has a magnification of $0.5$ and has a $f=100$ mm lens. 
Having camera images from two perpendicular axes allows for spatial overlapping in three dimensional space, more about this in \cref{subsec:Overlap}.

The top windows of the glass cell is used by the identical Mitutoyo microscope objective. 
The objective is brought in focus with the tweezers and the MOT is overlapped with both microscope objectives, ensuring we are in focus.


A picture showing the laser-induced fluorescence from the MOT as imaged through the camera on the side is shown in \cref{fig:LiF}.
A Lorentzian fit of the spatial profile in $x$ and $y$-directions is shown in \cref{fig:LiF}.
When moving the MOT with the compensation coils, we see the MOT seem to change shape. 
Probably, this is due to the compensation coils not being perfectly Helmholtz (their spacing is unequal to their radius). 
As a result, the Helmholtz pair that makes a field in the $y$-direction could also have $x$ and $z$ components, for example. We think that for us this is not really a problem, however.

\begin{figure}
    \centering
    \includegraphics[width=0.65\textwidth]{figures/FluoresenceAndFits.pdf}
    \caption{A picture of laser-induced fluorescence from the magneto-optical trap as imaged onto side camera. 
    The colors denote the relative intensity.
    Also, a Gaussian fit is shown in the $x$ and $y$-directions.}
    \label{fig:LiF}
\end{figure}

   
We estimate the number of captured atoms in the MOT by counting the number of counts on our side camera, subtracting background counts by turning off the field by summing over all pixels in a region of interest around the MOT $\sum_j p_j$.
We keep in mind the exposure time $\tau_s$, which varied but was typically $10$ ms, camera gain $G = 1$ and sensitivity $C$.
The camera sensitivity constant $C$ for this particular camera was determined to be $6 \cdot 10^3$ counts per photon and was calibrated using a variety of ND filters and a laser at 780 nm of known intensity. The total number of atoms is now
 
 \begin{equation}\label{eq:AtomNumber}
     N = \left( \frac{2}{\gamma\beta}\right)
     \left(\frac{4\pi l^2}{R^2}\right)
     \sum_j \frac{p_j}{\tau_s G C}.
 \end{equation}
In \cref{eq:AtomNumber} $\gamma = 2\pi \cdot 6.0 \cdot 10^6$ is the aforementioned linewidth of the D2 transition of Rb-85, $\beta \sim 0.6$ is a parameter taking into account photon loss of the glass cell and band-pass filter, $l = 20$ cm is the distance from the MOT to the collection lens and $R = 25$ mm is its radius. 
An additional factor of 2 comes in to account for the atoms occupying the excited state half of the time on average. 
We typically find a an atom number $\sim 10^5$.
This is a relatively small number of atoms for a Rb magneto-optical trap, but still orders of magnitude more than we need for loading them in arrays of optical tweezers. 
A possible explanation for the rather small number of atoms is that the unbalance in the various MOT beams: as a result of the rather large angle the vertical MOT beams make with the glass cell to make room for the microscope objective(s), there is significant loss of power in the retro-reflected beam, which has to pass the glass cell twice. 
In a next iteration of this Rb machine or a Sr equivalent, it would be useful to have 6 independent laser beams instead of 3 retro-reflected ones, though this comes at a cost of requiring more power. 
But the quantity that is of more relevance for optical tweezers is the MOT temperature, which was measured by Rik van Herk to be $200$ $\mu$K \cite{Herk2022}, which is slightly above the Doppler temperature. 
 

\section{Tweezer Arrays}\label{sec:Tweezers}

The next step in the optical setup is the optical tweezer array. 
The optical setup used for the optical tweezer array was already introduced in \cref{fig:TiSandSLMsetup,fig:SLMbeampath}b.
The implementation of this setup in the machine was already shown in \cref{fig:GlassCellTop}.
Using two sets of relay-mirrors, the beam from the \ac{SLM} is steered to the mirror sending the beam in vertical direction to the microscope objective (see \cref{fig:GlassCellSide}. 
In reality, from the top view in\cref{fig:GlassCellTop} the horizontal MOT beam and tweezer beam overlap which each other, therefore the tweezer beam is shown under a slight angle here.
A polarizing beam splitter cube is added which can be used to add additional laser beams to send into the objective, e.g. Rydberg lasers or laser beams for re-arrangement of the optical tweezer array. 
At the moment, this cube is unused, but we added it because it will change the optical path length of the SLM-objective path.

The microscope objective is again positioned on the same 5-axis piezo stage, using a custom aluminum holder.
To allow for arbitrary positioning of the stage with respect to the glass cell, it is not mounted directly into the vertical breadboard but in a set of base-plates that in turn screw into the table.
The holder is shown in \cref{fig:Coils}.

\begin{figure}
	\begin{subfigure}{.49\linewidth}
		\flushleft
		\includegraphics[height=7.6cm]{figures/CoilsCropped.jpg}
		\caption{}
		\label{fig:Coils1}
	\end{subfigure}
	\hfill
	\begin{subfigure}{.49\linewidth}
		\flushright
		\includegraphics[height=7.6cm]{figures/CoilsCropped2.jpg}
		\caption{}
		\label{fig:Coils2}
	\end{subfigure}
	\caption{
	\textsf{\textbf{a)}} and  Housing for the magnetic field coils. The glass cell is moved into the coils. 
    The hole on the right is for the horizontal MOT beam, while th angled beams travel into the section with the large holes. The microscope objectives are positioned on 5-axis piezo controlled stages using a custom aluminum mount. 
    \textsf{\textbf{b)}} Picture from another angle, showing the piezo stage for the top objective on top and the vacuum vessel in the background as well. 
    }
    \label{fig:Coils}
\end{figure}


\subsection{Overlap with MOT}\label{subsec:Overlap}

To load the tweezers with ultra-cold atoms, it is to be overlapped with the magneto-optical trap.
The most straightforward method is to move the MOT in 3D-space using additional bias coils.
To this end, a magnetic field coil holder was designed by Deon Janse van Renseburg and Rik van Herk. 
It is shown in \cref{fig:Coils} and features a set of Anti-Helmholtz field coils producing a quadrupole magnetic as well as a set of coils in the directions perpendicular to it.
The coils perpendicular to the anti-Helmholtz coils are not exactly Helmholtz because of space limitations, more about this in the thesis of Rik van Herk \cite{Herk2022}.
We found moving the MOT to work very nicely over a range of $\sim 1$ cm, which is more than plenty in overlapping with the \ac{ODT}.
We did notice however, the MOT changing shape for larger bias fields. 
Probably, the magnetic field gradient is not fully linear.
This should pose no issue, however.

To ensure we are overlapped with the ODT, we bring the \ac{Ti:S} laser producing the tweezer pattern to the resonance frequency of \ac{Rb}.
This will make the tweezer visible on the top camera as it no longer blocked by the band-pass filter.
On this top camera, we can overlap the \ac{MOT} in the horizontal direction, and move the MOT vertically until we are in focus. 
Because the frequency is now resonant, and the scattering force becomes important, we can also verify we are destroying the MOT by off-balancing it, which we can see on the horizontal camera in \cref{fig:GlassCellSide}.
Using this method: one camera for 2D overlap and one camera for the vertical overlap, we verify we are overlapped in 3D space. 

The power supply used to drive the coils\footnote{KORAD KC3405 programmable DC power supply} as well as the \ac{AOM}s are controlled by a custom Python control program written by Deon Janse van Renseburg, with a graphical user interface in \textit{PyQt}.


\section{Imaging Tweezer Arrays}\label{sec:TweezerImaging}

As discussed in \cref{sec:ArraysResults}, we have two methods of seeing our tweezers.
Both methods should be possible with our current setup and we plan to use them both, combining the best of both worlds.
Contrary to the method in \cref{ch:tweezer}, information will be lost because the imaging resolution will be similar to the resolution limit of the tweezers. 

\subsection{Direct Imaging}

This is similar to the method in \cref{ch:tweezer}. 
But, because the glass cell has a outer thickness of 30 mm, we need an additional ultra-long working distance glass-thickness compensated objective.
The microscope objective has a working distance when imaging through 3.5 mm N-BK7 glass of 15.08 mm, but we know that the glass cell has 4 mm thick quartz glass.
As a result, the paraxial rays focus a distance $d(1-1/n)$ further, where $d$ is the excess thickness and $n$ the refractive index of NBK-7.
Marginal rays focus further, but are corrected for using the spherical aberration pattern. 
Thus, we estimate the working distance in conjunction with this glass is $15.18$ mm. 
Thus, it should be possible to get a second objective in focus, but this requires positioning the objective less than $0.2$ mm close to the glass cell.
The tweezer beam coming out of the second objective can be focused onto a linear CCD camera. 
The direct tweezer image can be used to verify the array remains aberration-free during the experiment. 
Additionally, it can be used to dynamically adapt the weight factors in the \ac{GSW} algorithm. 

We did not manage to get a parallel beam out of the top microscope objective: shown in \cref{fig:GlassCellSide} the objective forms a focus, which we use another achromatic lens for to correct. 
The objective is designed for parallel operation and will thus contain aberration. 
The working distance of the objective with 3.5 mm cover glass is 15.08 mm. 
Because of the extra glass with thickness $d_{ex} = 0.5/n$ mm, we estimate the new working distance to be $15.08 + d_{\text{ex}}(1-1/n) \approx 15.19$ mm.
This should be just enough to get both objectives in focus because the outer thickness of the glass cell is 30 mm, but even if the glass cell is almost touching we did not get a parallel beam. 
For now, it is not a problem the top camera is slightly abberated, but for the next iteration of the machine we ordered a new glass cell from Japan Cell with 25 mm outer thickness, so this should no longer be a problem anymore.  

\subsection{Fluorescence Imaging}

For actual useful cold-atom experiments, we want to see if we have actual atoms in the tweezer. 
The way to do this is to use atomic fluorescence.
If the atom in the tweezer sees near-resonant light, it can scatter light releases atomic fluorescence. 
Because this fluorescence is typically fairly weak, we should maximize the amount of scattered photons that we can actually collect. 
The fraction $\eta$ of fluorescence that can be captured by our imaging system is 

\begin{equation}\label{eq:Collection}
    \eta = \frac{\Omega}{4\pi} = 
    \frac{1}{2}\left(1-\sqrt{1-\frac{\text{NA}^2}{n^2}}\right)
\end{equation}
For $\textsf{NA}=0.5$ and $n=1$, this collection fraction is $\eta =6.7$\%. 
Because the amount of photons that can be collected from an atom in a tweezer is limited, it is important to maximize \cref{eq:Collection}.
We collect the laser-induced fluorescence using the same microscope objective as used for producing the tweezer, separating it from the tweezer beam using a \ac{DM} \footnote{Thorlabs DMLP805, 1 inch}.
This is shown in \cref{fig:GlassCellTop}.
This dichroic mirror has a 805 cut-off, which is why 820 nm was chosen for the dipole traps.
We could collect the fluorescence from the top objective as well, but this has the disadvantage that the top objective is looking directly in the dipole laser, which we found near impossible to fully separate from the relatively weak fluorescence signal. 

The fluorescence is steered onto the Andor Zyla 5.5 sCMOS camera chip covered by a 780 band-pass filte, see \cref{fig:GlassCellTop,fig:GlassCellSide}.
The pixel size is 6.5 $\mu$m, or 19.5 $\mu$m using $3\times3$ pixel binning. 
It is focused onto the chip using a $f= 55$ mm lens\footnote{Nikon Micro-NIKKOR 55 mm f/2.8}.
Using \cref{eq:InfinityMagnification} this comes down to a magnification factor of $13.75$.
Assuming our spots are $0.8$ $\mu$m in $1/e^2$ radius, which after convolution with the imaging optics is roughly $1.1$ $\mu$m.
With the $13.75$ magnification, it turns out $3\times 3$ binning is probably the best choice for us. 
Thus, the spots are imaged onto one $3\times3$ binned chip when aligned with the center of the spot.
By trying to map the tweezer spots one to one on the camera, the signal to noise ratio is maximized. 

We focus the sCMOS camera by adjusting the $f=55$ mm field lens.
We do not have a feature we can focus onto, but we know we should be focused at infinity to be in focus with our tweezer array. 
We ensure we focus at infinity by shining a laser source collimated by a fiber outcoupler directly onto our camera, adjusting the focus until the parallel is imaged onto the chip as a diffraction-limited spot. 


The near-resonant light we will refer to as the probe light. 
In principle, the MOT beams could be used for this, as they should be near-resonant as well. 
However, we were worried using the MOT beams would cause too much noise, as they have fairly high power we say quite a large amount of scattering in the vacuum chamber on the camera, especially from the angled beams, which can internally reflect in the glass cell.
To also control over the exact power, detuning as well as in which beam the probe light goes, we used a separate laser path for this. 
As shown in \cref{fig:RbLaserSetup}, we split off a tiny amount of power to a separate AOM, denoted as probe AOM. 
The probe is recombined with the MOT beams in \cref{fig:GlassCellTop}, but we have full control over which beam it will go because of the $\lambdaup/2$ plate.
We typically run the probe with a blue detuning of $\delta \sim +3\gamma$ (the transition is Stark shifted to the blue) and intensity $I = 0.5 I_{\text{sat}}$, which for our beam diameters comes down to $\sim0.$5 mW per beam. 

\subsection{Experimental Sequence}

After letting the MOT build up for about 800 ms, we overlap the tweezer with it for about 150 ms.
This is done by a mechanical shutter, positioned on the \ac{Ti:S} laser table.
The delay of the shutter is in the order of a couple ms. 
During imaging, the MOT beams should be turned off, because its scattering will make the dipole traps impossible to see. 
The MOT beams can be shuttered off using the \ac{AOM}s in a matter of micro seconds. 
The MOT magnetic field will take longer to shut off due to Eddy currents in the coils. 
Rik van Herk measured a $1/e$ decay time of 8 ms, so we typically wait about 20 ms after turning off the MOT with the imaging. 
The imaging consists of triggering the probe beam as well as triggering the sCMOS camera at the same time. 
The triggering sequences are fed to the hardware using a programmable pattern generator custom built in our group. 
The imaging sequence is shown in \cref{fig:Sequence}.

For the atomic fluorescence pictures shown, the magnetic field was left on, but in the future it should be turned off. 
The reason is that the probe light is circularly polarized, which will cause Zeeman energy shifts in the array (\cref{eq:DetuningFull}).  
Due to Eddy currents, the magnetic field cannot be instantly switched off. 
Rik van measured measured a $1/e$ decay time of 8 ms when turning off the field \cite{Herk2022}, which in our case is definitely useable 

\begin{figure}
    \centering
    \includegraphics[width=0.76\textwidth]{figures/Sequence.pdf}
    \caption{Experimental sequence used to image the tweezer arrays. The entire sequence takes about 1.7 seconds. 
    The optical dipole trap overlapped for 150 ms with the magneto optical trap. 
    The probe and camera shutter start 10 ms after shutting off the MOT beams.}
    \label{fig:Sequence}
\end{figure}

\subsection{Fluorescence Images}

Below we show several images of atoms in our tweezer. 
These images were recorded using the aforementioned experimental sequence. 
We also performed exactly the same experimental sequence, but without turning no the dipole trap laser. 
The latter should only yield background as a result of atoms remaining from the expanding MOT or scattered photons from the probe beam. 
We subtracted the two images from each other, yielding the image in ...
This procedure was averaged a number of times as the signal to noise ratio was too low when not averaging. 
It is unclear why this is the case but we think these are some of the most important ones

\begin{enumerate}
    \item Low tweezer loading rate. 
    \item Camera has high amount of noise. 
\end{enumerate}

Starting with 1. The MOT has 




Because the quantum-efficiency and readout noise of this camera is sub-optimal for single-atom detection, we will upgrade to the Andor iXon Ultra 888 in the future. 





