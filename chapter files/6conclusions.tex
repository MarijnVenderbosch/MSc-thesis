\section{Conclusion}

To conclude, we have built a new apparatus that traps and images Rb atoms in an array of optical tweezers. 
We studied the light distribution from the glass-compensated microscope objective when using an input beam $\lambdaup = 820$ from a titanium sapphire laser with Gaussian waist equal to the aperture radius, finding a near diffraction-limited waist of $0.79 \pm 0.05$ $\mu$m after deconvolution with the 0.85 NA imaging objective. 
From diffraction theory, we know that the ideal result would be a 9\% broader waist than the $0.689$ $\mu$m Abbe limit because of the Gaussian input beam used, which is $\approx 0.75$ $\mu$m.
We studied the axial direction by scanning the microscope objective along the optical axis, finding a Rayleigh range of $6.2\pm0.5$ $\mu$m, which is much larger than the result from diffraction theory at $\sim 3.0$ $\mu$m for our parameters (beam diameter, numerical aperture). 
After aberration correction using the SLM we find a Rayleigh range of $3.6 \pm 0.5$ $\mu$m.
The aberrations originated from the incorrect cover glass thickness, which is effectively a spherical aberration as well as astigmatism caused by the SLM.
Most likely, the latter is caused by reflecting from the SLM under an angle.
These aberrations did not significantly affect the intensity distribution in the beam waist (radial direction).
The trap depths of the tweezer array showed a non-homogeneity of $\sim 7-12$ \%.
Larger arrays are typically less uniform, probably because the SLM patterns contain higher spatial frequencies for more complicated patterns, introducing deviations. 

The MOT produces roughly $\sim10^5$ atoms at a temperature of a of $\sim220$ $\mu$K.
Because the temperature is relatively hot, we need large trap depths to reliable trap atoms in tweezers, requiring large amounts of laser power limiting the amount of achievable traps. 
The fluorescence recorded from the neutral atom array shows a low signal to noise ratio.
The noise could originate from probe light scattering in the vacuum chamber or from the camera itself. 


\section{Outlook}

\begin{itemize}
    \item In order to improve the signal to noise ratio, we could perform spatial filtering, blocking out light that is not originating from the focus of the objective.
    Also, we can shutter the camera, eliminating the problem of the charge not clearing correctly. 
    But the noise of the camera itself (clock induced charge, readout noise) will probably still be high. 
    Therefore, we bought a new camera: the Andor iXon Ultra 888.
    This is a electron-multiplying \ac{CCD}, which has higher sensitivity, making them ideal for imaging single atoms in tweezer arrays.
    Additionally, this new camera should also significantly increase our quantum efficiency.
    
    \item If our signal to noise is improved and we no longer have to average a large number of images, we can study the average number loaded in our tweezer array in a quantitative fashion.
    To see whether we indeed load single atoms, as a function of time the photon count from a tweezer site should show behavior that is 'binary'. A count number above a certain threshold would correspond to a single atom, where counts below this threshold means the tweezer is empty or zero atoms. 
    
    \item If we indeed have single atoms and the average amount of atoms in a tweezer is $\simeq 0.5$, this would still mean that on average the array will only be half-filled, which is undesirable if the atoms are to interact with each other when excited to Rydberg states. 
    To solve this, an additional sorting beam could be used. 
    This sorting beam is an single optical tweezer potential, overlapped with the array. 
    The sorting beam can be moved quickly using a set of two \ac{AOD}s in crossed configuration.
    Because its potential is deeper than the tweezer potential, it can be used to re-arrange the atoms one by one as shown by \cite{Barredo2016}.
    In each experimental sequence, the sites of the array that are occupied are recorded. 
    An algorithm computes on the fly the optimum moves to generate a defect-free array, and the end result is verified using another fluorescence image. 
    
    \item To scale to a higher number of tweezer (qubits), the temperature of the MOT atoms should decrease, which would also lower the trap depth needed. 
    For Rb, sub-Doppler cooling techniques exist such as polarization-gradient cooling. 
    Sr, on the other hand, has the advantage that we can use the narrow-linewidth red MOT, which would open up the road to cooling to 1 $\mu$K temperatures.
    This opens up the road to higher fidelities, less noise and larger neutral atom arrays. 
    Also, with its two ground states, Sr is more suitable for quantum computation applications.
    
    \item To induce interactions between neutral atoms, a laser that excites to Rydberg states will have to be implemented. 
    Our group has done this on Rb before \cite{Bijnen2013}, but for Sr the implementation is entirely different. 
    The exact Rydberg level we want to use has not been decided upon yet. 
\end{itemize}
