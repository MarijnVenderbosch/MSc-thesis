For a quantum computer, multiple qubits are needed that are capable of interacting with each other. Thus, we need to make an array of the tweezers described in \cref{ch:tweezer}, spaced from each other in the order of micrometers. This can be done by sending different laser beams to the objective, all under slightly different angles. Experimentally, this can be realized by an \ac{AOD}: a device that diffracts light off of a sound wave in a crystal, where the degree of diffraction is controlled by a RF signal. By using 2 AODs in an orthogonal configuration, 2D Arrays can be realized \cite{Manuel2016}.



This chapter is on another method of making arrays of tweezers: the spatial light modulator. This device prints a hologram onto a laser beam, such that the interference pattern from this hologram in the focal plane of a lens yields any arbitrary array of spots. Spatial light modulators are thought to enable scaling to higher number of tweezers because when correctly used it can improve uniformity. In principle it is even possible to go make 3D arrays. 

\section{The Spatial Light Modulator}

The \ac{SLM} can manipulate properties of light like amplitude, phase and polarization. The type of SLM used is the phase-only SLM, and as the name suggests it can only control the phase of the light field. It does this by deploying a large array of pixels, where each individual pixel has a liquid crytal in it. The orientation of the crystal will change the refractive index because of the birefringence effect. The pixelated display is manufactured on a layer of silicon to address the voltage of each cell invididually. A sketch of the SLM pixels is shown in \cref{fig:LCoS}.

The working principle of this type of SLM has been documented extensively within our group, for example by \cite{Dijk2012,Bijnen2013,Bijnen2015}. 

\begin{figure}
\centering
	\begin{subfigure}{.4\textwidth}
		\centering
		\includegraphics[height=3.8cm]{figures/crossAOD.pdf}
		\caption{}
		\label{fig:CrossAOD}
	\end{subfigure}
	\begin{subfigure}{.5\textwidth}
		\centering
		\includegraphics[height=3.8cm]{figures/LCoS.png}
		\caption{}
		\label{fig:LCoS}
	\end{subfigure}
	\caption{a) Two AODs in crossed configuration, used to make a 2D spot array. Figure from \cite{Madjarov2021}. b) The orientation of the liquid crystal cells changes as a function of the applied electric field $E$. Figure from \cite{Guzman2017}.}
	\label{fig:GerschbergSaxton}
\end{figure}


\section{Phase Modulation}

In a phase only \ac{SLM}, each pixel can locally retard the field. As a result of diffraction, one can create arbitrary interference patterns in in the focal plane of the lens as shown by \cite{Bijnen2013}. The birefingence effect means the \ac{SLM} has two different refractive indices, somewhat confusingly called the ordinary and extraordinary refractive indeces respectively ($n_o$ and $n_e$). If the liquid crystal has a thickness $t$, the phase retardation will be $\phi_o = k t n_o$ and $\phi_e(V) = k t n_e(V)$ respectively, such that the phase retarder can be represented by the Jones matrix \cite{Guzman2017}

\begin{equation}\label{eq:JonesMatrix}
    M = e^{i \phi_0} 
    \begin{pmatrix}
        e^{i(\phi_e-\phi_o)} & 0\\
        0 & 1
    \end{pmatrix}
\end{equation}

Thus the phase retardance $\phi$ as a function of the applied voltage is \cite{Guzman2017}

\begin{equation}
    \phi(V) = k (n_e - n_0) t = \frac{2\pi}{\lambda} \Delta n t,
\end{equation}

where we absorbed the difference in refractive index in between the ordinary and extraordinary refractive indices in the paramter $\Delta n$.
This proces can be done separately for each pixel, leading to a phase retardance $
\phi(x',y')$ where $x',y'$ are the coordinates in the plane of the SLM. When a laser beam with intensity $|U_i|^2$ impinges on the SLM, it will pick up a phase. Thus its field can be described as $U_i e^{i\phi(x',y')}$ this is shown in \cref{fig:SLMgeometry}. The field will interfere with itself, until at infinity (or in the focal plane of a lens) the distribution evolves to $|U_f|^2$. \cref{sec:PropagationDerivation,sec:GSW} delve deeper in how this quantitatively works. First we will note that the refractive index is not linear in the applied voltage, we thus need to calibrate the SLM first. 

\subsection{Calibration}

The retardiation is a result of the liquid crystal molecules rotating in their cells. However, the retardation will not be linear as a function of the applied field $E$ and thus the voltage $V$. To linearize this, we need to calibrate the electro-optic response of the \ac{SLM}, which equates to measuring the retardation $\phi$ as a function of the applied voltage. The light field will oscillate with frequencies in the THz regime thus measuring this phase shift directly is not practical. Instead, we use a method that measures the amount of power in the first diffraction order known as the diffractive method. 

There are a variety of methods to calibrate a phase-only \ac{SLM} \cite{Li2019}. We chose the diffractive calibration method, originally proposed by \cite{Zhang1994}. We imprint a Ronchi grating onto the SLM, see \cref{fig:RonchiGrating}. Consisting of alternating grey values $L_1$ and $L_2$. This grating consists of $M$ periods $w$ in the horizontal direction and a vertical height $H$. Assuming amplitude modulation between different grey levels is negilible and the SLM only modulates phase, the electric field just after the SLM is propertional to the phase transmittance given by

\begin{equation}\label{eq:FieldAfterSLM}
    f(x,y) = \sum_{m=0}^{M-1} \left\{
    \operatorname{rect}\left(\frac{x-m w - w/4}{w/2}\right) + e^{i \phi(L)} \operatorname{rect}\left(\frac{x - m w - 3 w/4}{w/2}\right)
    \right\}
\end{equation}

According to Fourier optics, if we place a lens after the SLM, the field in the \textit{Fourier} plane of the lens is the Fourier transform of the field after the SLM, as derived in \cref{eq:2Dcase}. The intensity is thus for $y=0$

\begin{equation}\label{eq:FourierIntensity}
    |F(p,0)|^2=
    \frac{M^2 w^2}{2}\operatorname{sinc}^2\left(\frac{M f_x w}{2}\right) \cdot
    \frac{1 + \cos{\left[\phi(L)+f_x w/2\right]}}{\cos^2(f_x w/4)}
\end{equation}

Such that the intensity in the first diffraction order is

\begin{equation}\label{eq:IntensityFirstOrder}
    I_1(\phi) =
    \frac{8M^2w^2}{\pi^2} \left( 
    1-\cos{(\phi(L))}
    \right)
\end{equation}

\begin{figure}
	\begin{subfigure}{.54\linewidth}
		\includegraphics[height=4cm]{figures/LUTcalibrationSetup.pdf}
		\caption{}
		\label{fig:RonchiGrating}
	\end{subfigure}
	\hfill
	\begin{subfigure}{.44\linewidth}
		\includegraphics[height=4cm]{figures/lut_plot.pdf}
		\caption{}
		\label{fig:LUTcalibration}
	\end{subfigure}
	\caption{a) We put a Ronchi grating onto the SLM with grey values $L_1$ and $L_2$ We seperate the diffraction orders using a $f=400$ mm lens and block all orders other than the $+1$ order using an iris. b) Normalized power in the first order as a (\textcolor{red}{red}) and the fitted voltage response (\textcolor{blue}{blue}) as a function of the applied grey level $L_1 \in (0-255)$.}
	\label{fig:GerschbergSaxton}
\end{figure}

Experimentally, we load a sequence of Ronchi gratings onto the SLM, looping $L_1$ from the minumum to the maximum grey value, while keeping $L_2$ consant. For our 8-bit SLM, this is 0-255. For each hologram, seperate the multiple diffraction orders using a longer focal length lens (such that the spacing between the various orders is sufficient) and an aperture and record the power in the first order using a power meter. The results are shown in \cref{eq:LUTcalibration}. Also shown is the result of fitting \cref{eq:IntensityFirstOrder}: the voltage response as a fuction of grey level. This is also called the gamma curve or \ac{LUT}. The LUT is the 8-bit grey level to 10-bit voltage levels of the SLM, such that the electro-optic response is linear. As can be seen from \cref{fig:LUTcalibration}, the obtained LUT is rather non-linear.

The LUT calibration has to be performed because each individual SLM has a slightly different electro-optic response. Also, it is wavelength specific, which is why we performed it for the wavelenghts that we want to use for our optical tweezers. 

\section{Light Propagation}\label{sec:PropagationDerivation}

Finding the a\ac{CGH} to be displayed onto the SLM is not trivial. We will now explain how this is done. We consider the general case for 3D spot arrays, but will see that for 2D arrays a faster algorithm can be used. We consider an ensemble consisting of the SLM and a perfect lens with focal length $f$. We define two cartesian coordinate systems: the SLM plane by $\{x',y'\}$ and the focal plane of the lens by $\{x,y\}$. The situation is sketched in \cref{fig:SLMLens}

We adapt the following notation, the \ac{SLM} plane, with individual pixels indexed by $j$, such that the coordinates are $x'_j, y'_j,0$ ($z'=0)$ because by definition the pixels lie in this plane. Next, we place a lens with focal length $f$ one focal length away from the SLM plane. One focal plane further is the focal plane of this Fourier lens, which we call the Fourier plane. See \cref{fig:SLMgeometry}. 

\begin{figure}
    \centering
    \includegraphics[width = 12cm]{figures/SLMfigure.pdf}
    \caption{Field with intensity distribution $|U_i|^2$ impinging on the SLM, which due to its finite size acts as an aperture. The lens makes the resulting image $|U_f|^2$ in its focal plane. Also shown: two cartesian coordinate systems in the SLM and focal plane. Not to scale. Adapted from \cite{Labuhn2016}.}
    \label{fig:SLMLens}
\end{figure}

When no phase is applied onto the SLM, the lens will focus all the light down to a diffraction-limited in the Fourier plane origin ($x=y=z=0$). As result of traveling from the $j$th pixel to the $m$th trap, under paraxial approximation, the light picks up a phase along the way of 

\begin{equation}\label{eq:PropagationPhase}
    \Delta_j^m = \frac{2\pi z_m}{\lambdaup f^2}(x_j^2+y_j^2)+\frac{2\pi}{\lambdaup f}(x_j x_m +y_j y_m)
\end{equation}

This situation is sketched in \cref{fig:SLMgeometry}. In the SLM plane, we assume uniform illumination of the SLM such that the amplitude is the same $|U_i|$ everywhere. This may look like an unrealistic assumption considering we use a gaussian input beam, but we will see in a moment that in fact the input beam does not matter for the pattern produced. The SLM imprints a phase $\phi(x',y') = \phi_j$, such that the complex amplitude in the SLM plane for pixel $j$ is

\begin{equation}
    U_j(x',y') = U_j e^{i \phi_j}
\end{equation}

The contribution for the $m$th trap can be found by summing over all pixels, also keeping track of a phase factor in what is known as the diffraction formula:

\begin{equation}\label{eq:DiffractionFormula}
    \nu_m = e^{i k \left(2 f + z_m\right)}
    \frac{d^2}{i \lambda f} \sum_{j=1}^N e^{i(\phi_j - \Delta_j^m)}
\end{equation}

We assumed uniform illumination of the \ac{SLM}: $U_j =1$ everywhere. Furthermore $d$ is the pixel pitch of the SLM. But we are not interested in the phase of each in every spot, but only in the intensity. We will thus emit the prefactors in \cref{eq:DiffractionFormula} and normalize over $N$ pixels

\begin{equation}\label{eq:Vm}
    V_m = \frac{1}{N} \sum_{j} e^{i(\phi_j - \Delta_j^m)}
\end{equation}

A subset of 3D spot arrays are to make 2D arrays. Setting $z=0$ in \cref{eq:Vm} yields 

\begin{equation}\label{eq:2Dcase}
    V_m = \frac{1}{N} \sum_j \exp{\left(i\phi_j\right)} \exp{\left(
    - i 2\pi \left[
    \frac{x_m}{\lambdaup f} x_j + \frac{y_m}{\lambdaup f} y_j
    \right]
    \right)}
\end{equation}

Which is the 2D \ac{DFT} of $e^{i\phi}$ evaluated at spatial frequencies $x_m/\lambdaup f$ and $y_m/\lambdaup f$ in $x$ and $y$ respectively \cite{DiLeonardo2007,Bijnen2015}. The DFT can be efficiently evaluated on a computer using a \ac{FFT}. We stress that the FFT can only be used for 2D arrays. In this work, we only use 2D arrays and could in principle use \cref{eq:DFT} to speed up calculation. We did not do this however, as we only have to calculate the \ac{CGH} once so we do not mind it taking some more time. Making any arbitrary pattern in 2D, so not only spot arrays, is treated in detail in \cite{Bijnen2013}.  

\section{Finding the Hologram}\label{sec:GSW}

\cref{eq:Vm} gives the amplitudes of an array of spots given an array of phases $\phi_j$, but we are interested in the reverse problem: what phase $\phi_j$ should we apply, such that we have the spot intensities $|V_m|^2$? Because the equations describing light progagation are time-symmetric, this phase will be the result of $N$ coherent light sources radiating with $w_m e^{i \theta_m}$, picking up a propagation phase described by \cref{eq:PropagationPhase}, leading to a complex amplitude in the \ac{SLM} plane of 

\begin{equation}\label{eq:InterferencePattern}
    U_j (x',y') = \sum_m w_m \exp{
    i\left(\Delta_j^m + \theta_j\right)
    }
\end{equation}

But requires doing phase as well as amplitude modulation, when the SLM can only do phase, which is to take the argument of \cref{eq:InterferencePattern}

\begin{equation}\label{eq:Argument}
    \phi_j = \text{arg}\left\{
     \sum_m w_m \exp{
    i\left(\Delta_j^m + \theta_j\right)
    }
    \right\}
\end{equation}

This problem was solved by \cite{Gerschberg1972} and extended to 3D spot arrays by \cite{DiLeonardo2007}. The idea is to make use of an algorithm that virtually propagates light back and forth between the SLM and focal planes. For spot arrays, we are interested in the target intensities. Summing all of them yields the real part of the interference pattern, or the diffraction efficiency $e$ as well as the uniformity $u$

\begin{equation}
    e = \sum_m I_m = \sum_m |V_m|^2, 
    \quad 
    u = 1-\frac{\text{max}(I_m)-\text{min}(I_m)}{\text{max}(I_m)+\text{min}(I_m)}
\end{equation}

The algorithm should converge yielding a set $\{w_m, \phi_j\}$. The iteration counter is $k$

\begin{itemize}
    \item Starting out with an initial guess $\phi_0$ using random phases for each pixel $\theta_j$ and uniform $w_m$, calculate $V_m$ according to \cref{eq:Vm}. 
    
    \item For In the focal plane, replace the calculated $V_m^k$ array by the target $V_m$ array, and using updated $w_m^k = w_m^{k-1} V_m^k / |V_m^k|$ find the new phase mask $\phi_j^k$ according to \cref{eq:InterferencePattern}.
\end{itemize}

This sequence is sketched in \cref{fig:GerschbergSaxton}. Starting with phase mask $\phi_j^k$ we use the diffraction equation \cref{eq:DiffractionFormula} to compute the complex amplitudes of the spot array $V_m^{k+1}$. From the amplitudes, we find new weight factors $w_m$ and pixel values $\theta_m$, which are used to compute the new phase mask using the interference equation \cref{eq:InterferencePattern}. Then, the iteration counter $k$ is increased, the phase mask is updated and the procedure repeats. 

This type of algorithm typically converges after a couple tens of steps. The algorithm was implemented in Python by Ivo Knottnerus as part of his PhD research. Because the SLM has some $\sim 2$ million pixels, converting to and from the SLM plane, so performing equations  \cref{eq:Vm,eq:Argument} can be time consuming depending on the desired pattern. Therefore, these computations are performed onto the GPU using the \textit{PyOpenCL} library. 

\begin{figure}
\centering
	\begin{subfigure}{.56\textwidth}
		\centering
		\includegraphics[height=5cm]{figures/SLMgeometry.pdf}
		\caption{}
		\label{fig:SLMgeometry}
	\end{subfigure}
	\begin{subfigure}{.43\textwidth}
		\centering
		\includegraphics[height=5cm]{figures/WeightedGerschbergSaxton.pdf}
		\caption{}
		\label{fig:MOTconcept}
	\end{subfigure}
	\caption{a) SLM plane with phases $\phi_j(x,',y',z=0)$ for pixel $j$, as well as the focal plane or Fourier plane with trap coordinates for trap $m$ of $(x_m,y_m,z_m)$. Figure from \cite{DiLeonardo2007}. b) The \ac{GSW} algorithm visualized. The light field is virtually propagated between the SLM and focal planes using \cref{eq:Vm,eq:Argument}. Each iteration, the weight factors $w_m$ and SLM phases $\theta_m$ are updated. }
	\label{fig:GerschbergSaxton}
\end{figure}


\section{Operating the SLM}

The algorithm as explained in \cref{IFTA} was implented in software by \cite{Bijnen2015}. The input intensity $|PU_0|^2$ is assumed to be a plane wave. Using the SLM is explained here. The SLM will inevitably lose power due to several processes. These are explained here. 

\subsection{Diffraction}

Diffraction. The SLM is a device consisting of individual pixels. Therefore, it is essentially a 2D diffraction grating, and light will diffract in several directions. In 1D, we know that according to diffraction theory the diffraction angle maxima $\theta_m$ from a plane wave incident at $\theta_i$ occur at $\theta_m = \arcsin(\sin\theta_i - m\lambdaup/d)$, where $m$ is the order of diffraction and $d$ the pixel pitch. We are interested in the light in the $m=0$ diffracton order, other orders are spatially filtered out and unused.

\subsection{Zeroth Order}

Light can reflect on the spacings between the pixels. Also, it can reflect off of the electrodes at the back of the device. All of this undiffracted light is focussed onto the optical axis by the objective. As a result, the optical axis unusable to shape light and we have to seperate our modulated light from this 'zeroth order peak'. This is done by superimposing a linear phase on top of the SLM phase pattern. Because the phase is computed modulo 2$\pi$ the resulting pattern is a blazed grating. Light from the higher diffraction orders and zeroth order peak can be blocked by an iris in the focal plane of an intermediaty lens. 
    
\subsection{Finite aperture size}

In order to employ the maximum amount of pixels the SLM has to offer, and therefore the maximum drawing area in the focal plane of the Fourier lens, all of the pixels should be illuminated. However, because the incident beam is described by a gaussian, this will lead to power loss as a result of light not falling on the active pixel area. 
    
We can estimate this power loss by shining a Gaussian beam $G(x,y)$ of width $w(z)$ and power $P_0$ onto a rectangular aperture of dimensions $(2S_x, 2S_y)$ where $S_{x,y}$ are the semi-widths of the aperture. The relative power transmission $P/P_0$ can be found by integrating the intensity of the beam \cref{GaussianBeamIntensity} in cartesian coordinates over the aperture

\begin{equation}\label{RectAperturePower}
    \frac{P}{P_0} =
    \iint I(x,y) dA=
    \text{Erf}\left(\frac{\sqrt{2}S_x}{w(z)}\right) \text{Erf}\left(\frac{\sqrt{2}S_y}{w(z)}\right)
\end{equation}

where Erf($\cdot$) denotes the error function. The optimum incident beam size $w(z)$ is thus a compromize between drawing area and power efficiency. We chose $w(z) \approx S_{x}$, specifically $w(z) = 4.8$ mm and $S_x = 4.9$ mm. 

\subsection{Reflectivity}

As can be seen in \cref{fig:LCoS}, the light will back reflect at after passing through the liquid crystal. This back plate will have a non-perfect anti relection coating, leading to some light being absorbed. 


\section{Calibration}

We know that the phase pattern of the SLM is realized by the birefringence effect. However, it turns out the phase retardation as a function of the applied voltage is non-linear. Additionally, the surface of the SLM is not fully flat, introducing a small aberration. Both of these effects differ for indiviual SLM units, even for the same model. Therefore, here we will briefly review how to calibrate the SLM to account for these two effect.s 

\subsection{Electro-Optic Response}

The goal of the calibration of the phase as a function of applied voltage (electro-optic response) is to 

\begin{enumerate}
    \item Achieve a linear electric-optic response. 
    \item Ensure the phase is modulate by a full wave $2\pi$ spanning the minimum to the maximum grey level. 
\end{enumerate}

\subsection{Optical flatness correction}

Because of the silicon producetion process, the chip is not completely flat. The optical flatness of SLM was measured to be $0.18\lambdaup$ by manufacturer Meadowlark. Additionally, the manufacturer measured the shape of the correction using Michelson interferometry. By substracting this measured non-flatness from the calculated phase pattern, this effect can be corrected. 


\section{Arrays of tweezers}

\subsection{Array Spacing}

Because of the properties of the Fourier transform, the smallest possible 

\subsection{Spot Detection}

We make an arbitrarily sized array of optical tweezers. For detecting maxima locations, we could simply set pixels above a certain count threshold as a maxima. The disadvantage of this is that a noisy pixel could be detected as a spot, which would hinder analysis. 

To combat this, we first convolute the image with a Laplacian of Gaussian filter, which first smoothes the image to reduce noise. Subsequently it computes the 2D laplacian to detect edges. If this second derivative is above a certain threshold, we detect the edge as a spot. If the detected amount of spots is not the expected amount, we can easily tune the threshold. The camera image is cropped into regions of interest around the spots marked by the Laplacian of Gaussian filter. 

\subsection{Spot Fitting}

Our tweezers should be described by \cref{FourierBesselAperture}. However, because of a lack of an analytic expression this is not the most convenient. \cref{FourierBesselAperture} can closely be resembled by a Gaussian. We will therefore use the following function for fitting spots numbered $k$

\begin{equation}\label{2DGaussian}
    G_k(x,y) = U_k \exp{\left(-\frac{(x_k-x_{k,0})^2}{2\sigma_{k,x}^2}\right)}
    \exp{\left( \frac{-(y_k-y_{k,0})^2}{2\sigma_{k,y}^2} \right)}
\end{equation}

Here $U$ is the 'trap depth', $x,y$ are the cartesian coordinates in the camera frame and $x_0,y_0$ is the center of the spot. Lastly $\sigma_{x,y}$ are standard deviations in $x$ and $y$. In principle, because the limiting aperture in the system is a circular aperture, the spots should be more or less symmetric and $\sigma_x\approx \sigma_y$ which we confirmed. We find the $1/e^2$ waist $w_k=\sigma_{k,x}+\sigma_{k,y}$. 