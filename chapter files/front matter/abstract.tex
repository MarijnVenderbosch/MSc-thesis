\noindent We report on the progress towards a neutral-atom based quantum co-processor.
As a first step towards this goal, we constructed a new apparatus that produces a magneto-optical trap in an optically contacted glass cell. 
We load $\sim 10^5$ Rb atoms in the magneto-optical trap of $T\sim200$ $\mu$K.
Subsequently, the atoms are loaded in an array of optical tweezers made by a spatial light modulator (Meadowlark, $1920 \times 1200$) and a glass-compensated ultra-long working distance microscope objective (Mitutoyo, 0.5 NA) at $\lambdaup = 820$ nm.
We measured the light distributions \textit{ex-situ} directly using a 0.85 NA objective, finding a near-diffraction limited waist of $0.79\pm0.05$ $\mu$m (theory: 0.75 $\mu$m) in the radial direction and a Rayleigh range of $6.2 \pm 0.5$ $\mu$m (diffraction theory: 3.0 $\mu$m) in the axial direction.
After aberration correction using the spatial-light modulator we find a corrected Rayleigh range $3.6\pm0.5$ $\mu$m.
The tweezer arrays light distributions showed a typical non-homogeneity of $\sim 7-12\%$.
Larger arrays are typically less uniform than smaller ones. 
\textit{In-situ}, ultra-cold atoms loaded from the magneto-optical trap in the tweezer array are imaged into a sCMOS camera (Zyla, Andor) though the same objective used for the tweezers by splitting off the atomic fluorescence using a long pass dichroic mirror.
